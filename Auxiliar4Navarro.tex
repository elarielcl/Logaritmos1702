\documentclass[dcc,uchile]{fcfmcourse}
\usepackage{teoria}
\usepackage[utf8x]{inputenc}
\usepackage{amsmath, amssymb, amsthm}
\usepackage{amsfonts,setspace}
\usepackage{listings}
\usepackage{hyperref}
\usepackage{color}
\usepackage{csquotes}
\usepackage{soul}
\usepackage{emojis}
\usepackage[linesnumbered,lined,boxed,commentsnumbered]{algorithm2e}
\renewcommand{\algorithmcfname}{Algoritmo}

%Teoremas, Lemas, etc.
\theoremstyle{plain}
\newtheorem{teo}{Teorema}
\newtheorem{lem}{Lema}
\newtheorem{prop}{Proposici\'on}
\newtheorem{cor}{Corolario}
\theoremstyle{definition}
\newtheorem{defi}{Definici\'on}
\newtheorem{obs}{Observaci\'on}
\newtheorem{ej}{Ejemplo}
\newtheorem{ejer}{Ejercicio}

\definecolor{pblue}{rgb}{0.13,0.13,1}
\definecolor{pgreen}{rgb}{0,0.5,0}
\definecolor{porange}{rgb}{0.9,0.5,0}
\definecolor{pgrey}{rgb}{0.46,0.45,0.48}

\lstset{language=Java,
  showspaces=false,
  showtabs=false,
  breaklines=true,
  showstringspaces=false,
  breakatwhitespace=true,
  commentstyle=\color{porange},
  keywordstyle=\color{pblue},
  stringstyle=\color{pgreen},
  basicstyle=\ttfamily,
  moredelim=[il][\textcolor{pgrey}]{$ $},
  moredelim=[is][\textcolor{pgrey}]{\%\%}{\%\%}
}

\newenvironment{codebox} {\small \ttfamily \obeylines \begingroup \setstretch{-2.4}} {\endgroup}

% COmpletar titulo
\title{Auxiliar 4 - ``Análisis Amortizado"}
\course[CC4102]{Diseño y Análisis de Algoritmos}
\professor{Gonzalo Navarro}
\assistant{\textst{Manuel} Ariel Cáceres Reyes}


\begin{document}
\maketitle
\begin{center}
11 de Septiembre del 2017
\end{center}


\vspace{-1ex}

\cfoot{``Your goal should be to pay off your credit card bills in full at the end of each month and set aside money toward your emergency savings''\\Suze Orman}

\begin{problems}
\problem {\large\underline{\textbf{Pila con MULTIPOP}}}\\
Considere una implementación de pila con sus operaciones de \texttt{POP} y \texttt{PUSH} a costo $\mathcal{O}(1)$. Además considere la operación \texttt{MULTIPOP(k)} que remueve los primeros $k$ elementos del tope de la pila y retorna el último de ellos, si la pila tiene $n<k$ elementos entonces se queda vacía. Esta operación tiene costo $\mathcal{O}(\min(n, k))$.\\
Muestre que el costo amortizado de las operaciones es $\mathcal{O}(1)$ (partiendo de pila vacía).
\problem {\large\underline{\textbf{Cola con 2 Pilas}}}\\
Con 2 Pilas implemente una Cola costos amortizados de \texttt{enqueue} y \texttt{dequeue} son de $\mathcal{O}(1)$.
\problem {\large\underline{\textbf{Dinamizando la Búsqueda Binaria}}}\\
Hacer \textit{búsqueda binaria} sobre un arreglo ordenado toma tiempo \textit{logarítmico} en el tamaño del arreglo, sin embargo la inserción de un nuevo elemento es \textit{lineal} pues se deben ``empujar'' elementos para mantener el arreglo ordenado.
\begin{enumerate}[a)]
    \item Diseñe una estructura de datos que permita hacer búsqueda en $\mathcal{O}\left((\log n)^2\right)$ e inserción en $\mathcal{O}(\log n)$ amortizado.
    \item Explique como se podría implementar la eliminación de elementos a costo $\mathcal{O}\left(n\right)$. Muestre además que con esto se pierde el $\mathcal{O}(\log n)$ amortizado de la inserción.
\end{enumerate}
\problem {\large\underline{\textbf{Árboles $\alpha$-balanceados}}}\\
Para un valor de $\alpha \in [0.5, 1)$, decimos que un nodo $x$ de un árbol binario es $\alpha$-balanceado si cumple que:
\begin{align*}
    |left(x)| &\le \alpha |x|\quad\\
    & \land\\
    |right(x)| &\le \alpha |x|&
\end{align*}
donde $|x|$ es el número de nodos del árbol con raíz en $x$ y $right(x), left(x)$ son los nodos a la derecha e izquierda de $x$ en el árbol. Decimos que un árbol binario de búsqueda es $\alpha$-balanceado si todos sus nodos son $\alpha$-balanceado.\\
Para $\alpha>\frac{1}{2}$ muestre como implementar un árbol binario de búsqueda $A$ $\alpha$-balanceado cuyos costos de inserción y eliminación sean amortizados \textit{logarítmicos}.\\

\textbf{HINTS}: Sea lazy, estudie las alturas de estos árboles, estudie los $1/2$-balanceados y utilice el potencial $\phi = \frac{1}{2\alpha-1}\sum_{x\in A; \delta(x) > 1} \delta(x)$, con $\delta(x) = ||right(x)|-|left(x)||$
\end{problems}
\end{document}
