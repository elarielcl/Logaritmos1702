\documentclass[dcc,uchile]{fcfmcourse}
\usepackage{teoria}
\usepackage[utf8x]{inputenc}
\usepackage{amsmath, amssymb, amsthm}
\usepackage{amsfonts,setspace}
\usepackage{listings}
\usepackage{hyperref}
\usepackage{color}
\usepackage{csquotes}
\usepackage{soul}
\usepackage{emojis}
\usepackage[linesnumbered,lined,boxed,commentsnumbered]{algorithm2e}
\renewcommand{\algorithmcfname}{Algoritmo}

%Teoremas, Lemas, etc.
\theoremstyle{plain}
\newtheorem{teo}{Teorema}
\newtheorem{lem}{Lema}
\newtheorem{prop}{Proposici\'on}
\newtheorem{cor}{Corolario}
\theoremstyle{definition}
\newtheorem{defi}{Definici\'on}
\newtheorem{obs}{Observaci\'on}
\newtheorem{ej}{Ejemplo}
\newtheorem{ejer}{Ejercicio}

\definecolor{pblue}{rgb}{0.13,0.13,1}
\definecolor{pgreen}{rgb}{0,0.5,0}
\definecolor{porange}{rgb}{0.9,0.5,0}
\definecolor{pgrey}{rgb}{0.46,0.45,0.48}

\lstset{language=Java,
  showspaces=false,
  showtabs=false,
  breaklines=true,
  showstringspaces=false,
  breakatwhitespace=true,
  commentstyle=\color{porange},
  keywordstyle=\color{pblue},
  stringstyle=\color{pgreen},
  basicstyle=\ttfamily,
  moredelim=[il][\textcolor{pgrey}]{$ $},
  moredelim=[is][\textcolor{pgrey}]{\%\%}{\%\%}
}

\newenvironment{codebox} {\small \ttfamily \obeylines \begingroup \setstretch{-2.4}} {\endgroup}

% COmpletar titulo
\title{Auxiliar 4 - ``Análisis Amortizado"}
\course[CC4102]{Diseño y Análisis de Algoritmos}
\professor{Gonzalo Navarro}
\assistant{\textst{Manuel} Ariel Cáceres Reyes}


\begin{document}
\maketitle
\begin{center}
11 de Septiembre del 2017
\end{center}


\vspace{-1ex}

\cfoot{``Your goal should be to pay off your credit card bills in full at the end of each month and set aside money toward your emergency savings''\\Suze Orman}

\begin{problems}
\problem {\large\underline{\textbf{Pila con MULTIPOP}}}\\
Considere una implementación de pila con sus operaciones de \texttt{POP} y \texttt{PUSH} a costo $\mathcal{O}(1)$. Además considere la operación \texttt{MULTIPOP(k)} que remueve los primeros $k$ elementos del tope de la pila y retorna el último de ellos, si la pila tiene $n<k$ elementos entonces se queda vacía. Esta operación tiene costo $\mathcal{O}(\min(n, k))$.\\
Muestre que el costo amortizado de las operaciones es $\mathcal{O}(1)$ (partiendo de pila vacía).
\problem {\large\underline{\textbf{Cola con 2 Pilas}}}\\
Con 2 Pilas implemente una Cola costos amortizados de \texttt{enqueue} y \texttt{dequeue} son de $\mathcal{O}(1)$.
\problem {\large\underline{\textbf{Dinamizando la Búsqueda Binaria}}}\\
Hacer \textit{búsqueda binaria} sobre un arreglo ordenado toma tiempo \textit{logarítmico} en el tamaño del arreglo, sin embargo la inserción de un nuevo elemento es \textit{lineal} pues se deben ``empujar'' elementos para mantener el arreglo ordenado.
\begin{enumerate}[a)]
    \item Diseñe una estructura de datos que permita hacer búsqueda en $\mathcal{O}\left((\log n)^2\right)$ e inserción en $\mathcal{O}(\log n)$ amortizado.
    \item Explique como se podría implementar la eliminación de elementos a costo $\mathcal{O}\left(n\right)$. Muestre además que con esto se pierde el $\mathcal{O}(\log n)$ amortizado de la inserción.
\end{enumerate}
\problem {\large\underline{\textbf{Árboles $\alpha$-balanceados}}}\\
Para un valor de $\alpha \in [0.5, 1)$, decimos que un nodo $x$ de un árbol binario es $\alpha$-balanceado si cumple que:
\begin{align*}
    |left(x)| &\le \alpha |x|\quad\\
    & \land\\
    |right(x)| &\le \alpha |x|&
\end{align*}
donde $|x|$ es el número de nodos del árbol con raíz en $x$ y $right(x), left(x)$ son los nodos a la derecha e izquierda de $x$ en el árbol. Decimos que un árbol binario de búsqueda es $\alpha$-balanceado si todos sus nodos son $\alpha$-balanceado.\\
Para $\alpha>\frac{1}{2}$ muestre como implementar un árbol binario de búsqueda $A$ $\alpha$-balanceado cuyos costos de inserción y eliminación sean amortizados \textit{logarítmicos}.\\

\textbf{HINTS}: Sea lazy, estudie las alturas de estos árboles, estudie los $1/2$-balanceados y utilice el potencial $\phi = \frac{1}{2\alpha-1}\sum_{x\in A; \delta(x) > 1} \delta(x)$, con $\delta(x) = ||right(x)|-|left(x)||$
\end{problems}
\newpage
\begin{center}
{\huge \underline{Soluciones}}
\end{center}
\begin{problems}
\problem Posible implementación de \texttt{MULTIPOP(k)}.
\begin{algorithm}
\SetKwProg{Fn}{Funcion}{}{}
\Fn{MULTIPOP (k)}{
\For{$i \gets 1\ldots k-1$}{
    $x\gets$ POP()\\
    \If{estaVacia()}{
        \Return $x$
    }
    
}\Return POP()}
\end{algorithm}
\\Notar que este procedimiento tiene complejidad $\mathcal{O}(\min(n,k))$ y que esto en el peor caso es  $\mathcal{O}(n)$. Veremos a continuación que alcanza complejidad amortizada $\mathcal{O}(1)$, en un conjunto de operaciones \texttt{POP}, \texttt{PUSH} y \texttt{MULTIPOP(k)}.
\paragraph{Análisis completo.} Supongamos que se hicieron $r$ operaciones de \texttt{MULTIPOP} cada una con costo proporcional a $k_{1}, k_{2}, \ldots, k_{r}$ respectivamente. Para poder haber hecho estas operaciones tienen que haber ocurrido al menos (en algún momento) $\sum_{i=1}^r k_{i}$ \texttt{PUSH} (para poder hacer esos \texttt{MULTIPOP}. Esto significa que en una secuencia de operaciones:
\begin{align*}
    C_{total} &= C_{PUSHs} + C_{POPs} + C_{MULTIPOPs} \\
    &\le 2C_{PUSHs} + C_{POPs}\\
    &\le 3|numero\ de\ operaciones|
\end{align*}
y por lo tanto el costo amortizado de las operaciones es $3$.

\paragraph{Contabilidad de Costos.} Si le cobramos al \texttt{PUSH} 1 por la inserción en el la pila del elemento y 1 por su posible posterior salida de la pila, las operaciones de \texttt{POP} y \texttt{MULTIPOP(k)} quedan de costo cero, teniendo $\mathcal{O}(1)$ por operación.

\paragraph{Potencial.}
Si definimos el potencial como $\phi = |Pila|$, tenemos que $\phi_0 = 0$ y además $\phi \ge 0$, entonces $\phi_i\ge \phi_0$, cumpliéndose la condición necesaria para ser un potencial válido.\\
Usando este potencial, el costo amortizado de \texttt{PUSH} queda en $\hat{c} = 1 + 1 = 2$, el de \texttt{POP} en $\hat{c} = 1 - 1 = 0$ y el de \texttt{MULTIPOP(k)} en $\hat{c} = \min(n,k) - \min(n,k) = 0$.
\problem Tendremos una pila de $entrada$ en donde \texttt{pusheamos} los elementos cuando ingresan a la cola, y además una pila de $salida$ de donde se sacan los elementos de la cola, en caso que esta última no tenga elementos se vacía la cola de $entrada$ en la de $salida$.
\begin{figure}[h]
    \centering
    \includegraphics[scale=0.5]{imagenes/dos_pilas}
\end{figure}

En el caso del \texttt{enqueue} es de costo $\mathcal{O}(1)$, sin embargo el \texttt{dequeue} es de costo $\mathcal{O}(n)$ pues puede potencialmente ocurrir un vaciado. Veremos a continuación que el costo amortizado de estas operaciones (partiendo de una cola vacía) es $\mathcal{O}(1)$.

\paragraph{Contabilidad de Costos.} Al ser encolado, un elemento paga por la operación de \texttt{PUSH} en la pila de $entrada$ y también paga por su posible posterior traslado de pila (un \texttt{PUSH} y un \texttt{POP}) por lo que su costo amortizado es de $\hat{c} = 3$. Por lo anterior, el costo de \texttt{dequeue} solo es el de hacer \texttt{POP} del elemento de la cola de $salida$ (pues si hay vaciado, el costo de traslado de pila ya fue pagado por el \texttt{enqueue}).

\paragraph{Potencial.} 
Si definimos el potencial como $2|Pila_{entrada}|$, entonces:
\begin{itemize}
    \item El \texttt{enqueue}, $\hat{c} = 1 + 2 = 3$
    \item El \texttt{dequeue},
    \begin{itemize}
        \item Sin vaciado, $\hat{c} = 1 - 2\cdot 0 = 1$
        \item Con vaciado, $\hat{c} = 1 + 2|Pila_{entrada}| - 2|Pila_{entrada}| = 1$
    \end{itemize} 
\end{itemize}
\problem
\begin{enumerate}[a)]
    \item Tendremos arreglos $A_{0}, A_{1}, A_{2}, \ldots, A_{\lceil \log n \rceil-1}$  de tamaños $2^0, 2^1, 2^2, \ldots, 2^{\lceil \log n \rceil-1}$ respectivamente, cada uno de los cuales estará o bien lleno de elementos del conjunto o bien vacío. Además estos arreglos estarán ordenados internamente, pero no existirá alguna relación de orden entre ellos (necesariamente). Si $a_{k}a_{k-1}\ldots a_{1}a_{0}$ es la representación binaria de $n$, los arreglos que estarán llenos serán aquellos $A_i$ que tengan $a_{i}$ en 1 y el resto estará vacío. Notar que esto asegura que tenemos los $n$ elementos en la estructura.\\
    
    Para buscar un elemento hacemos una búsqueda binaria en cada uno de estos arreglos. Como son $\lceil \log n \rceil$ de ellos, y cada uno es de tamaño $\le n$ esta búsqueda nos tomará $\mathcal{O}((\log n)^2)$.\\
    
    Para insertar un elemento buscamos el primer $a_{i} = 0$ y hacemos la unión de los arreglos $A_{<i}$ con el elemento a insertar, en el arreglo $A_{i}$, lo que se puede hacer a costo $\mathcal{O}(2^i)$ que en el peor caso puede ser $\mathcal{O}(n)$.
    
    \paragraph{Análisis Completo.} Veamos que para una secuencia de $m$ inserciones, el arreglo $A_i$ será unido $\frac{m}{2^{i+1}}$ veces y por lo tanto el costo agregado de las uniones será 
    \begin{align*}
        \sum_{i=0}^{\lceil \log n \rceil - 1} \frac{m}{2^{i+1}} \cdot 2^i = \lceil \log n \rceil \frac{m}{2}
     \end{align*}
    y por lo tanto el costo amortizado será $\mathcal{O}(\log n)$.
    \item Para eliminar, buscamos el primer $a_{i} = 1$. Luego buscamos el elemento que queremos eliminar (supongamos que no está en $A_{i}$) y lo intercambiamos con alguno de $A_i$. Finalmente dividimos $A_i$ en $A_0, A_1, \ldots, A_{i-1}$ a costo $\mathcal{O}(2^i)$ potencialmente $\mathcal{O}(n)$. Si consideramos esta eliminación y la inserción antes descrita obtenemos el siguiente conjunto de operaciones:
    \begin{enumerate}[1.]
        \item Insertar elementos hasta que tengamos todo en $|A_i| = 2^i = n$
        \item Eliminar un elemento a costo $n$
        \item Insertar un elemento a costo $n$
        \item Repetir 2,3 tanto como sea necesario para que el costo total sea $\sim n^2$
    \end{enumerate}
    y de este modo las operaciones no pueden tener costo $o(n)$ real y tampoco amortizado.
\end{enumerate}
\problem 
\paragraph{Lazy} Insertaremos/Eliminaremos como en un Árbol de Búsqueda Binaria(ABB), y luego convertimos el nodo más alto que no es $\alpha$-balanceado en $1/2$-balanceado (el subárbol completo). Notar que ser $\beta$-balanceado implica ser $\alpha$-balanceado cuando $\beta\le\alpha$, por lo que la inserción anterior es correcta.
\paragraph{Estudio de la altura de $\alpha$-balanceado} Notemos que la altura del árbol con $n$ nodos, en el peor caso, será uno más que la altura de un árbol de $\alpha n$ nodos (pues esta es la máxima cantidad de nodos que puede tener alguno de los subárboles hijos). Finalmente resolviendo la ecuación de recurrencia ($h(n) = 1 + h(\alpha n)$) obtenemos que $h(n) \in \mathcal{O}(\log_{1/\alpha} n) = \mathcal{O}(\log n)$. Por lo que la búsqueda en estos árboles es logarítmica (y la inserción y eliminación sin reconstrucción).

\paragraph{Estudio de los $1/2$-balanceados} Para construir un árbol $1/2$-balanceado a partir de un ABB cualquiera con raíz $x$ podemos hacer un recorrido \textit{inorden} para obtener los elementos ordenados , luego recursivamente ir eligiendo la mediana del arreglo como raíz (en $\mathcal{O}(|x|)$). El resultado de lo descrito anteriormente es un árbol que cumple que para todo nodo $x$, $\delta(x)\le 1$.

\paragraph{Proposición}.
\\Un Árbol es $1/2$-balanceado sii todos los nodos del Árbol cumplen con $\delta(x)\le 1$\\

\underline{$\Rightarrow$}\\
Por contradicción, existe un nodo $x$ tal que $\delta(x)>1$. Supongamos sin pérdida de generalidad que $|left(x)| = |right(x)| - 2$, por lo tanto se cumple que $|right(x)| - 2 + |right(x)| = |x|-1$ y por lo tanto $2|right(x)| = |x|+1$, pero por ser $1/2$-balanceado se cumple que $|x|+1\ge 2|right(x)| + 1$ lo que es una contradicción.\\

\underline{$\Leftarrow$}\\
En caso que $\delta(x) = 0$ se cumple $|right(x)| = |left(x)| = \frac{|x|-1}{2} \le \frac{|x|}{2}$. Si en cambio $\delta(x) = 1$, sin pérdida de generalidad $|right(x)| = |left(x)| + 1$ que es igual a $|x| - |right(x)|$ y por lo tanto $|left(x)| < |right(x)| = \frac{|x|}{2}$
\end{problems}
 
 \paragraph{Análisis del potencial.}
 Con esto, la reconstrucción propuesta más arriba genera un árbol $1/2$-balanceado. Además, el potencial de un árbol $1/2$-balanceado es 0.\\
 
 La inserción y eliminación sin reconstrucción cambian $\delta(x)$ solo de los nodos en el camino de inserción y eliminación a lo más en 1 por cada nodo y por lo tanto el cambio de potencial es $\mathcal{O}(log n)$ obteniendo un costo amortizado de $\mathcal{O}(log n)$.\\
 
 Por otro lado, supongamos una reconstrucción sobre $x$ que tiene costo  $|x|$, el potencial luego de la reconstrucción es 0 (pues queda $1/2$-balanceado) y antes de la reconstrucción se cumple $\delta(x) = ||right(x)|-|left(x)|| \ge \alpha|x| - (|x| - \alpha |x|) = (2\alpha - 1) |x|$ y por lo tanto el cambio de potencial es mayor o igual a $|x|$ dejando el costo de la reconstrucción amortizado  $\mathcal{O}(1)$.
 
\end{document}
