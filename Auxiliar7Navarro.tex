\documentclass[dcc,uchile]{fcfmcourse}
\usepackage{teoria}
\usepackage[utf8x]{inputenc}
\usepackage{amsmath, amssymb, amsthm}
\usepackage{amsfonts,setspace}
\usepackage{listings}
\usepackage{hyperref}
\usepackage{color}
\usepackage{csquotes}
\usepackage{soul}
\usepackage{emojis}
\usepackage[linesnumbered,lined,boxed,commentsnumbered]{algorithm2e}
\renewcommand{\algorithmcfname}{Algoritmo}

%Teoremas, Lemas, etc.
\theoremstyle{plain}
\newtheorem{teo}{Teorema}
\newtheorem{lem}{Lema}
\newtheorem{prop}{Proposici\'on}
\newtheorem{cor}{Corolario}
\theoremstyle{definition}
\newtheorem{defi}{Definici\'on}
\newtheorem{obs}{Observaci\'on}
\newtheorem{ej}{Ejemplo}
\newtheorem{ejer}{Ejercicio}

\definecolor{pblue}{rgb}{0.13,0.13,1}
\definecolor{pgreen}{rgb}{0,0.5,0}
\definecolor{porange}{rgb}{0.9,0.5,0}
\definecolor{pgrey}{rgb}{0.46,0.45,0.48}

\lstset{language=Java,
  showspaces=false,
  showtabs=false,
  breaklines=true,
  showstringspaces=false,
  breakatwhitespace=true,
  commentstyle=\color{porange},
  keywordstyle=\color{pblue},
  stringstyle=\color{pgreen},
  basicstyle=\ttfamily,
  moredelim=[il][\textcolor{pgrey}]{$ $},
  moredelim=[is][\textcolor{pgrey}]{\%\%}{\%\%}
}

\newenvironment{codebox} {\small \ttfamily \obeylines \begingroup \setstretch{-2.4}} {\endgroup}

% COmpletar titulo
\title{Auxiliar 7 - ``Algoritmos Online, Dominios Discretos y Análisis Amortizado"}
\course[CC4102]{Diseño y Análisis de Algoritmos}
\professor{Gonzalo Navarro}
\assistant{\textst{Manuel} Ariel Cáceres Reyes}


\begin{document}
\maketitle
\begin{center}
16 de Octubre del 2017
\end{center}


\vspace{-1ex}

\cfoot{``Those who can imagine anything, can create the impossible.''\\Alan Turing}

\begin{problems}
\problem {\large\underline{\textbf{k servidores, online (continuación)}}}\\
Considere el escenario donde tiene $k$ puntos (\textit{servidores}) en un espacio \textit{métrico} y una secuencia de puntos (\textit{peticiones}) que debe atender. Cada vez que llega una petición, un servidor debe moverse hacia esa posición para atenderla.\\
El problema \textit{online} consiste en minimizar la distancia recorrida por todos los servidores luego de $n$ peticiones, sin saber la secuencia de puntos a atender.\\
Consideremos ahora el problema de $2$ servidores y sus peticiones en una línea y el siguiente algoritmo:
    \begin{itemize}
        \item Si ambos servidores están al mismo lado de la petición, se envía el servidor más cercano.
        \item Si la petición está entre los dos servidores, se envían los dos a la misma velocidad constante, deteniéndose cuando uno de ellos llega al objetivo.
    \end{itemize}
    Muestre que este algoritmo es 2-competitivo.
    \problem {\large\underline{\textbf{Números en rango}}}\\ Describa un algoritmo que, dados $n$ enteros en $[0, \ldots, k]$, preprocese su entrada y responda cuántos de estos enteros se encuentran en el rango $[a, \ldots, b]$ en tiempo constante. Su algoritmo debería tomar tiempo $\mathcal{O}(n + k)$ en el preprocesamiento.
\problem  {\large\underline{\textbf{Árbol cartesiano}}}\\
Dado un arreglo de enteros distintos $A[1, n]$, un árbol cartesiano es un árbol binario de $n$ nodos. Si el mínimo de $A$ está en $A[i]$, entonces la raíz del árbol cartesiano corresponde a $A[i]$, el hijo izquierdo al árbol cartesiano de $A[1, i − 1]$ y el hijo derecho al árbol cartesiano
de $A[i + 1, n]$. Cuando el rango de $A$ se hace vacío el árbol cartesiano es vacío.
\begin{enumerate}[a)]
    \item Dibuje el árbol cartesiano de $A[1, 10]$ = $[2, 9, 4, 6, 7, 5, 3, 1, 8, 10]$.
    \item Muestre que el árbol cartesiano se puede construir en tiempo $\mathcal{O}(n)$ a partir de $A[1, n]$. Para esto, considere usar inducción, procesando los elementos de $A$ de izquierda a derecha y análisis amortizado para obtener la cota.
\end{enumerate}
\end{problems}

\newpage
\begin{center}
{\huge \underline{Soluciones}}
\end{center}
\begin{problems}
\item Consideremos un conjunto de consultas en la línea. Para realizar este análisis compararemos el costo amortizado del ALG con el costo real de cada operación del OPT llegando así a la competitividad.\\
Para esto definamos lo siguiente:
\begin{itemize}
    \item $(s_{1})_{i}, (s_{2})_{i}$ como las posiciones de los servidores de ALG después de haber respondido a la $i$-ésima consulta.
    \item $(s_{1}^*)_{i}, (s_{2}^*)_{i}$ como las posiciones de los servidores de OPT después de haber respondido a la $i$-ésima consulta.
    \item $c_{i}$ es costo incurrido por ALG para responder la $i$-ésima consulta.
    \item $c^*_{i}$ es costo incurrido por OPT para responder la $i$-ésima consulta.
    \item $(m_{1})_{i}$, valor de uno de los emparejamientos entre los servidores de OPT y ALG luego de la $i$-ésima consulta. Corresponde a $d((s_{1})_{i},(s_{1}^*)_{i}) + d((s_{2})_{i},(s_{2}^*)_{i})$.
    \item $(m_{2})_{i}$, valor del otro emparejamiento entre los servidores de OPT y ALG luego de la $i$-ésima consulta. Corresponde a $d((s_{1})_{i},(s_{2}^*)_{i}) + d((s_{2})_{i},(s_{1}^*)_{i})$.
    \item $m_{i} = \min((m_{1})_{i}, (m_{2})_{i})$, valor del emparejamiento mínimo entre los servidores de OPT y ALG luego de la $i$-ésima consulta.
    \item $\phi_{i} = 2m_{i} + d((s_{1})_{i}, (s_{2})_{i}) = 2m_{i} + d_{i}$, es decir, el doble de emparejamiento mínimo más la distancia de los servidores de ALG.
\end{itemize}
Como los servidores comienzan en la mismas posiciones $\phi_{0} = 0$, y dado que el potencial es una suma de distancias $\phi_{i}\ge 0$, por lo que es un potencial válido.\\

Analicemos ahora el cambio de potencial de una consulta cualquiera.\\
Cuando OPT mueve sus servidores $c^*_{i}$, $d_{i}$ no cambia, y $m_{i}$ puede cambiar a lo más $c^*_{i}$, pues puedo encontrar el mismo emparejamiento mínimo que tenía antes (el cual puede ser o no mínimo ahora), el que vale a lo sumo $c^*_{i}$ unidades más. Con esto $\Delta \phi_{i} \le 2c^*_{i}$.\\
Cuando ALG mueve sus servidores $c_{i}$:
\begin{itemize}
    \item Si la consulta no cayó entre los servidores, entonces solo uno de ellos se mueve $c_{i}$ alejándose del otro. Por lo tanto $d_{i}$ aumenta en $c_{i}$.\\
    Sin perdida de mucha generalidad, podemos asumir que el servidor de la derecha va a atender la consulta, en donde se encuentra uno de los servidores de OPT, moviéndose $c_{i}$ hacia este. Dicho esto el otro servidor de ALG se encuentra a la izquierda del que responde la consulta y lo único que nos queda analizar son las posibles posiciones del otro servidor de OPT.
    \begin{itemize}
        \item A la derecha del servidor de OPT que respondió la consulta.
        \item Entre el servidor que respondió la consulta en OPT y el que lo hará en ALG.
        \item Entre los servidores de ALG.
        \item A la izquierda del servidor izquierdo de ALG.
    \end{itemize}
    En los $4$ casos se puede apreciar que $m_{i}$ disminuye $c_{i}$. Y por lo tanto ALG produce $\Delta \phi \le -c_{i}$.
    \item Si la consulta cayó entre los dos servidores de ALG, entonces $d_{i}$ disminuye en $c_{i}$. Por otro lado, como ambos se acercan $c_{i}/2$ a un servidor de $OPT$, uno de ellos se acerca esa distancia en el emparejamiento mínimo, mientras que el otro a lo más se aleja esa distancia por lo que el valor de $m_{i}$ no aumenta (aunque puede que se reduzca). Y también $\Delta \phi \le -c_{i}$.
\end{itemize}
Por lo tanto, $\Delta \phi_{i} \le 2 c^*_{i} - c_{i}$ y $\hat{c}_{i} \le 2c^*_{i}$. Y de este modo, $C_{ALG} = \sum c_{i} \le \sum \hat{c}_{i} \le \sum 2c^*_{i} = 2C_{OPT}$.
\item Si hacemos \textit{CountingSort} sobre los elementos en tiempo $\mathcal{O}(n+k)$ podemos obtener el arreglo $C$ que en $C[i]$ indica la posición del último $i$ del conjunto de $n$ enteros y por lo tanto, la cantidad de enteros $\le i$, además si hay algún $i \in \{1,\ldots k\}$ que no está en el conjunto de enteros entonces $C[i] = C[pred(i)]$ donde $pred(i)$ es el mayor entero menor a $i$ que si está en el conjunto. Con esto, el número de enteros en el rango $[a,\ldots,b]$ será $C[b]-C[a-1]$ \grin .
\item 
\begin{enumerate}[a)]
    \item Este es el árbol cartesiano correspondiente:
    \begin{figure}[h]
        \centering
        \includegraphics[scale=0.4]{imagenes/cartesiano}
    \end{figure}
    \item Construiremos un algoritmo que vaya generando el árbol cartesiano de un prefijo del arreglo que se nos da como input, es decir, si el arreglo de input es $A[1,n]$ entonces haremos una iteración de $n$ pasos y al final del paso $i$ tendremos el árbol cartesiano correspondiente a $A[1,i]$.\\

Además mantendremos el invariante de tener un puntero $p$ al hijo más derecho del árbol, esto, pues en el árbol cartesiano de $A[1,i]$ el hijo más derecho es precisamente $A[i]$ (pues es el último elemento).
\\

La construcción entonces es como sigue, empezamos con el árbol de un solo nodo (que seteamos como $p$) $A[1]$. Luego, asumiendo que tenemos el árbol cartesiano para $A[1,i]$ y $p$ apuntando al nodo $A[i]$, si $A[i+1] > A[i]$ entonces colgamos a $A[i+1]$ como hijo derecho de $A[i]$ y seteamos $p$ al nodo $A[i+1]$. En caso que $A[i+1] < A[i]$ subimos por el árbol desde $p$ hasta encontrar un nodo $f$ que contenga un $A[j] < A[i+1]$, si $s$ es su hijo derecho entonces ponemos a $A[i+1]$ como hijo derecho de $f$, ponemos $p$ en ese nodo y como su hijo izquierdo a $s$. Finalmente con esto tenemos el árbol cartesiano de $A[i+1]$ con $p$ apuntando a $A[i+1]$ que es el nodo más derecho.\\

Para mostrar que este algoritmo toma tiempo $\mathcal{O}(n)$ veremos que las operaciones toman tiempo amortizado $\mathcal{O}(1)$.\\

Si definimos el potencial $\phi_{i}$ como la profundidad de $p$ luego de la iteración $i$ entonces:

\begin{itemize}
\item Si estamos en el caso $A[i+1] > A[i]$, entonces el potencial aumenta en $1$, pero el costo real es constante por lo que el costo amortizado también lo es.
\item Si es el caso que $A[i+1] < A[i]$, y $A[j]$ esta a distancia $k$ de $p$ entonces el cambio de potencial es $-k+\Theta(1)$, pero por otro lado el costo es $k+\Theta(1)$, por lo que el costo amortizado resulta ser $\Theta(1)$.
\end{itemize}
\end{enumerate}
\end{problems}
\end{document}