\documentclass[dcc,uchile]{fcfmcourse}
\usepackage{teoria}
\usepackage[utf8x]{inputenc}
\usepackage{amsmath}
\usepackage{amsfonts,setspace}
\usepackage{caption}
\usepackage{listings}
\usepackage{hyperref}
\usepackage{color}
\usepackage{soul}
\usepackage{emojis}
\usepackage[spanish]{babel}
\usepackage[lined,boxed,commentsnumbered]{algorithm2e}
\SetKwInput{KwIn}{Input}
\renewcommand{\algorithmcfname}{Algoritmo}
\renewcommand{\figurename}{Figura}
\renewcommand{\tablename}{Tabla}


\definecolor{pblue}{rgb}{0.13,0.13,1}
\definecolor{pgreen}{rgb}{0,0.5,0}
\definecolor{porange}{rgb}{0.9,0.5,0}
\definecolor{pgrey}{rgb}{0.46,0.45,0.48}

\lstset{language=Java,
  showspaces=false,
  showtabs=false,
  breaklines=true,
  showstringspaces=false,
  breakatwhitespace=true,
  commentstyle=\color{porange},
  keywordstyle=\color{pblue},
  stringstyle=\color{pgreen},
  basicstyle=\ttfamily,
  moredelim=[il][\textcolor{pgrey}]{$ $},
  moredelim=[is][\textcolor{pgrey}]{\%\%}{\%\%}
}

\newenvironment{codebox} {\small \ttfamily \obeylines \begingroup \setstretch{-2.4}} {\endgroup}

% COmpletar titulo
\title{Tarea Recuperativa - Multiplicación de Matrices}
\course[CC4102]{Diseño y Análisis de Algoritmos}
\professor{Gonzalo Navarro}
\assistant{Manuel Cáceres}
\assistantt{Jaime Salas}
\assistantt{Tomás Perry}

\begin{document}
\captionsetup[table]{name=Tabla}
\captionsetup[table]{name=Figura}

\maketitle
\vspace{-1ex}
\section{Introducción}
El problema de \textit{Verificación de Multiplicación de Matrices} (VMM) es como sigue: dadas tres matrices de $n \times n, A, B, C$ de números enteros, queremos verificar si $A\cdot B = C$. Se pide implementar una solución probabilista para este problema y medir su desempeño. Se espera que el alumno implemente individualmente los algoritmos y entregue un informe que indique claramente los siguientes puntos:
\begin{enumerate}[1.]
    \item Las \textit{hipótesis} escogidas antes de realizar los experimentos.
    \item El \textit{diseño experimental}, incluyendo los detalles de la implementación de los algoritmos, la generación de las instancias y las medidas de rendimiento utilizadas.
    \item La \textit{presentación de los resultados} en forma de una descripción textual, tablas y/o gráficos.
    \item El \textit{análisis e interpretación} de los resultados.
\end{enumerate}
\section{Algoritmo Probabilista}
La solución que consideraremos es la siguiente, donde la entrada del algoritmo es $A, B, C$ matrices de $n\times n$ y $k\ge 1$ es un parámetro del algoritmo.
\begin{itemize}
    \item Repetir $k$ veces:
    \begin{itemize}
        \item Escoger vector $\vec{r}\in \{0,1\}^n$, donde cada coordenada es un bit escogido con al azar de manera independiente.
        \item Si $A\cdot \left(B\cdot \vec{r}\right) \not = C \cdot \vec{r}$, rechazamos.
    \end{itemize}
    \item Aceptamos.
\end{itemize}
Observe que si $A\cdot B$ es efectivamente $C$, entonces el algoritmo siempre acepta. Si $A\cdot B \not = C$ el algoritmo podría equivocarse y aceptar.

\section{Experimentos}
Para los experimentos genere dos matrices $A$ y $B$ de $n\times n$, para $n=1000$, con números enteros elegidos al azar e independientemente en el rango $[-10000, 10000]$. Compute $C = A\cdot B$.
\begin{itemize}
    \item Para cada $m= 1, \ldots, 10$, deberá alterar la matriz $C$ de la siguiente manera: escoja al azar $100\cdot m$ celdas de la matriz y modifique sus contenidos por nuevos números escogidos al azar y de manera independiente en el rango $[-10000, 10000]$.
    \item Dadas las matrices $A$, $B$ y la matriz $C$ modificada, para cada $k = 1, 2, 3 ,\ldots$ debe estimar el tiempo y la tasa de error del algoritmo con parámetro $k$ sobre la instancia $A,B,C$. En este caso la tasa de error corresponde a la fracción de ejecuciones en donde el algoritmo acepta. Considere $100$ ejecuciones en cada caso. Debe alcanzar un valor de $k$ tal que la tasa de error sea $0$. Si esto requiere un valor demasiado grande para $k$, indique su tasa de error alcanzada.
\end{itemize}
\section{Entrega de la Tarea}
\begin{itemize}
    \item La tarea es individual.
    \item Para la implementación puede utilizar \texttt{C}, \texttt{C++}, \texttt{Java} o \texttt{Python}. Para el informe se recomienda utilizar \LaTeX.
    \item Escriba un informe claro y conciso. Las ponderaciones del informe y la implementación en su nota final son las mismas.
\end{itemize}
\end{document}

